\documentclass[12pt,a4paper]{article}

\usepackage[russian]{babel}
\usepackage{amsmath}
\usepackage{amssymb}
\usepackage{mathtools}

\usepackage{cmap}
\usepackage{hyperref}

\usepackage{longtable}
\usepackage{setspace}
\usepackage{graphicx}

\newcommand{\eqdef}{\stackrel{\text{def}}{=}}
\newcommand{\xin}[3]{x_{#1} \in \left[ #2, #3 \right]}
\newcommand{\set}[1]{
    \left\{ \begin{matrix*}[l] #1 \end{matrix*}     
    \right\}   
    }


\singlespacing

\oddsidemargin=-1cm
\textwidth=18.0cm
\topmargin=-1.54cm%0pt
\textheight=24cm

%%%%%%%%%%%%%%%%%%% BEGINNING %%%%%%%%%%%%%%%%%%%%%%%%

\begin{document}

\begin{center}
    \large Координатные функции и локализующие множества
\end{center}
\begin{flushright}
    Корешков Василий, 15.12.19
\end{flushright}

{\large Система имеет следующий вид:}

\begin{equation}
    \left\{
    \begin{matrix*}[l]
    \dot{x}_1 & = & - \dfrac{\lambda_{c} x_{1} x_{4}}{K_{c} - x_{3} + 1} + x_{1}^{3/4} \left(1 - \sqrt[4]{\dfrac{x_{1}}{C_{0}}}\right) \left(k_{c} + \mu_{c} x_{2}\right)
    \\
    \dot{x}_2 & = & - \lambda_{p} x_{2} + x_{2} \left(1 - \dfrac{x_{2}}{P_{0}}\right) \left(k_{p} + \dfrac{\mu_{p} x_{1}}{K_{p} + x_{1}}\right)
    \\
    \dot{x}_3 & = & k_{r} - x_{3} \left(\gamma_{c} x_{1} + \gamma_{p} x_{2} + \lambda_{r}\right)
    \\
    \dot{x}_4 & = & \dfrac{k_{t} x_{3}}{K_{t} - x_{3} + 1} - \lambda_{t} x_{4}
    \end{matrix*}
    \right.
\end{equation}

\subsection{Координатные функции}

Рассмотрим координатные функции вида
\(\varphi_i = x_i\,,\; i = \overline{1,4}.\) Вернемся для этого к
уравнению \(F(x_1,x_2,x_3,x_4) = 0:\)


\begin{equation*}
    \left\{
        \begin{matrix*}[r]
            - \dfrac{\lambda_{c} x_{1} x_{4}}{K_{c} - x_{3} + 1} + x_{1}^{\frac{3}{4}} \left(1 - \sqrt[4]{\dfrac{x_{1}}{C_{0}}}\right) \left(k_{c} + \mu_{c} x_{2}\right) &=& 0
            \\
            - \lambda_{p} x_{2} + x_{2} \left(1 - \dfrac{x_{2}}{P_{0}}\right) \left(k_{p} + \dfrac{\mu_{p} x_{1}}{K_{p} + x_{1}}\right) &=&  0
            \\
            k_{r} - x_{3} \left(\gamma_{c} x_{1} + \gamma_{p} x_{2} + \lambda_{r}\right) &=& 0
            \\
            \dfrac{k_{t} x_{3}}{K_{t} - x_{3} + 1} - \lambda_{t} x_{4} &=& 0
        \end{matrix*}
    \right.
\end{equation*}

\begin{enumerate}
    
    \item Для первой координатной функции имеем:
    \[ S_{\varphi_1} = \{ x_1 = 0\}\, \cup \, 
    \left\{ x_1 = 
    \frac{C_{0} \alpha^{4}{\left(x_{2},x_{3} \right)}}{\left(\sqrt[4]{C_{0}} \lambda_{c} x_{4} + \alpha{\left(x_{2},x_{3} \right)}\right)^{4}}
    \right\} \]

    Явное выражение для $x_1$:

        
        $$ x_1 = \frac{C_{0} \alpha^{4}{\left(x_{2},x_{3} \right)}}{\left(\sqrt[4]{C_{0}} \lambda_{c} x_{4} + \alpha{\left(x_{2},x_{3} \right)}\right)^{4}},\, \text{где}\, \alpha = \left(k_{c} + \mu_{c} x_{2}\right) \left(K_{c} - x_{3} + 1\right) > 0$$

        
        $$\nabla \varphi_1 = \left(
            \begin{matrix}\dfrac{4 C_{0}^{\frac{5}{4}} \lambda_{c} \mu_{c} x_{4} \left(k_{c} + \mu_{c} x_{2}\right)^{3} \left(K_{c} - x_{3} + 1\right)^{4}}{\left(\sqrt[4]{C_{0}} \lambda_{c} x_{4} + \left(k_{c} + \mu_{c} x_{2}\right) \left(K_{c} - x_{3} + 1\right)\right)^{5}} 
            \\
            - \dfrac{4 C_{0}^{\frac{5}{4}} \lambda_{c} x_{4} \left(k_{c} + \mu_{c} x_{2}\right)^{4} \left(K_{c} - x_{3} + 1\right)^{3}}{\left(\sqrt[4]{C_{0}} \lambda_{c} x_{4} + \left(k_{c} + \mu_{c} x_{2}\right) \left(K_{c} - x_{3} + 1\right)\right)^{5}} 
            \\
            - \dfrac{4 C_{0}^{\frac{5}{4}} \lambda_{c} \left(k_{c} + \mu_{c} x_{2}\right)^{4} \left(K_{c} - x_{3} + 1\right)^{4}}{\left(\sqrt[4]{C_{0}} \lambda_{c} x_{4} + \left(k_{c} + \mu_{c} x_{2}\right) \left(K_{c} - x_{3} + 1\right)\right)^{5}}
            \end{matrix}\right)^T $$

    Отсюда видно, что градиент ни в какой точке не обращается в нуль, потому что любая компонента нигде не ноль, поскольку \(\alpha > 0.\)
    Конкретнее,
    \(\dfrac{\partial \varphi_1}{\partial x_2} > 0,\,  \dfrac{\partial \varphi_1}{\partial x_3} < 0,\,  \dfrac{\partial \varphi_1}{\partial x_4} < 0.\)
    Очевидно, \(\varphi_{1 \min} = 0\) (при \(x_1 = 0\)). Для установления
    супремума воспользуемся сохренением знаков частных производных: значение будет тем больше, чем меньше \(x_3\) и \(x_4\) и чем больше \(x_2\). Положив
    \(x_3 = x_4 = 0\), получим
    \(x_1 = C_0 \Rightarrow \varphi_{1 \max} = C_0\)

    \item Для второй координатной функции:
    \[ S_{\varphi_2} = \{ x_2 = 0\}\, \cup \, 
    \left\{ x_2 = 
    \dfrac{P_{0} \left(K_{p} k_{p} - K_{p} \lambda_{p} + x_{1} \left(k_{p} - \lambda_{p} + \mu_{p}\right)\right)}{K_{p} k_{p} + x_{1} \left(k_{p} + \mu_{p}\right)}
    \right\} \]
    
    Явное выражение для $x_2$:
    
        
        $$ x_2 = \frac{P_{0} \left(K_{p} k_{p} - K_{p} \lambda_{p} + x_{1} \left(k_{p} - \lambda_{p} + \mu_{p}\right)\right)}{K_{p} k_{p} + x_{1} \left(k_{p} + \mu_{p}\right)}$$
    
        
        $$ x_2' = - P_{0}\dfrac{ \left(\left(k_{p} + \mu_{p}\right) \left(K_{p} k_{p} - K_{p} \lambda_{p} + x_{1} \left(k_{p} - \lambda_{p} + \mu_{p}\right)\right) + \left(K_{p} k_{p} - x_{1} \left(k_{p} + \mu_{p}\right)\right) \left(k_{p} - \lambda_{p} + \mu_{p}\right)\right)}{\left(K_{p} k_{p} + x_{1} \left(k_{p} + \mu_{p}\right)\right)^{2}} $$
    
        
        А точнее, $ x_2' = K_p P_0  \dfrac{k_p\lambda_p + (k_p + \mu_p)(\lambda_p - 2k_p)}{\left( K_p  k_p + x_1 \left(k_p + \mu_p\right)\right)^{2}}  $
    
    Докажем теперь, что производная (при допустимых значениях параметров) всегда меньше нуля. Для этого рассмотрим неравенство:

    $$ x_2' < 0\; \Leftrightarrow \;
    k_p\lambda_p + (k_p + \mu_p)(\lambda_p - 2k_p) < 0\; \Leftrightarrow\;
    2 k_p \lambda_p + \mu_p\lambda_p < 2 k_p(k_p + \mu_p)
    \Leftrightarrow $$
    $$ \Leftrightarrow\; 
    \dfrac{\lambda_p}{k_p} < 2 \dfrac{k_p + \mu_p}{2k_p + \mu_p}.
    $$
    
    При этом известно, что $\lambda_p < k_p$. С другой стороны,
    $$ 2 \dfrac{k_p + \mu_p}{2k_p + \mu_p} = 
    1 + \dfrac{\mu_p}{2k_p + \mu_p} > 1.     
    $$

    Таким образом, данное неравенство выполняется при любых допустимых значениях параметров. Легко теперь получить $\varphi_{2 \max}$ и $\varphi_{2 \min}$:
    
    $\varphi_2(0) = \varphi_{2 \max} = P_0 \left( 1 - \frac{\lambda_p}{k_p}\right) > 0;\;  \varphi_2(+\!\infty) = 1 - \frac{\lambda_p}{k_p + \mu_p}.$
    Однако $ \varphi_{2 \min} = 0 $, так как $\{~x_2~=~0\}\,~\subset~S_{\varphi_2}$.


    \item Явное выражение для $x_3$:

        
        $$ x_3 = \frac{k_{r}}{\gamma_{c} x_{1} + \gamma_{p} x_{2} + \lambda_{r}}$$

        
        $$
        \nabla \varphi_3 = \left(\begin{matrix}- \dfrac{\gamma_{c} k_{r}}{\left(\gamma_{c} x_{1} + \gamma_{p} x_{2} + \lambda_{r}\right)^{2}} 
        &
        - \dfrac{\gamma_{p} k_{r}}{\left(\gamma_{c} x_{1} + \gamma_{p} x_{2} + \lambda_{r}\right)^{2}}\end{matrix}\right) 
        $$

        
    Очевидно, что никогда не выполняется \(\nabla x_3 = 0\,\) (раз все \(x\) неотрицательны), так что функция \(\varphi_3(\vec{x}) = x_3\) достигает своих экстремальных значений (на универсальном сечении) лишь на границе, либо не достигает их вовсе. Действительно, видно, что достигается максимум при \(x_1 = x_2 = 0,\) при этом \(\varphi_{3 \max} = \dfrac{k_r}{\lambda_r}\). А вот минимального значения \(\varphi_3\) не достигает, но \(\varphi_{3 \inf} = 0\), очевидно.

    \item Явное выражение для $x_4$:

        
        $$ x_4 = \frac{k_{t} x_{3}}{\lambda_{t} \left(K_{t} - x_{3} + 1\right)};$$

        
        $$x_4' = \dfrac{k_{t} x_{3}}{\lambda_{t} \left(K_{t} - x_{3} + 1\right)^{2}} + \dfrac{k_{t}}{\lambda_{t} \left(K_{t} - x_{3} + 1\right)} > 0.$$

        
        Приходим к выводу, что \(\varphi_{4 \min} = 0\) (достигается при
    \(x_3 = 0\)), а максимум достигается при $x_3 = 1,\varphi_{4 \max}
    = \dfrac{k_t}{\lambda_t K_t} $

\end{enumerate}

\subsection{Построение последовательности локализующих функций и множеств}

Мы смогли получить все 4 координатные функции. Каждая координатная
функция определяет, таким образом, соответствующее локализующее
множество \(\Omega_i\), полученное из \(\overline{\mathbb{R}_{+}^{4}}\)
следующим образом (далее всюду подразумеваются подмножества
\(\overline{\mathbb{R}_{+}^{4}}\)):

\begin{equation*}    
    \begin{matrix}
    \Omega_1 = \left\{
            x \;\big|\; x_1 \in \left[ 0, C_0  \right]
                \right\}; 
    &
    \Omega_2 = \left\{
            x \;\Big|\; x_2 \in \left[ 0, 
            P_0 \left( 1 - \frac{\lambda_p}{k_p}\right)  \right]  
            \right\};
    \\ \\
    \Omega_3 = \left\{
        x \;\Big|\; x_3 \in \left[ 0, \dfrac{k_r}{\lambda_r}  \right]
            \right\};
    &
    \Omega_4 = \left\{
        x \;\Big|\; x_4 \in \left[ 0, \dfrac{k_t}{\lambda_t K_t} \right]
            \right\}.
    \end{matrix}
\end{equation*}

Приступим к построению первых четырех элементов последовательности локализующих функций, а именно: $\{\varphi_2, \varphi_3, \varphi_4, \varphi_1\}$.

\begin{enumerate}
    
    \item \textbf{Первые 4 члена итерационной последовательности:}

    \begin{enumerate}
        
        \item Обозначим $ G_1^1 \eqdef \Omega_2 =
            \set{\xin{2}{0}{P_0\left( 1 - \dfrac{\lambda_p}{k_p} \right)}} $. 

        \item $\varphi_3$ на $G_1^1$ имеет те же экстримальные значения на универсальном сечении, что и в $\overline{\mathbb{R}_{+}^{4}}$, так что
        $$G_2^1 = G_1^1 \cap \Omega_3 = \set{
            \xin{2}{0}{P_0\left( 1 - \dfrac{\lambda_p}{k_p} \right)} \\
            \xin{3}{0}{\dfrac{k_r}{\lambda_r}}
        }.
        $$

        \item Теперь воспользуемся тем, что ${x_4}'_{x_3} > 0:$ 
        $$\varphi_{4 \sup}(G_2^1) = x_4(x_{3 \max}) = 
            \dfrac{k_t k_r}{\lambda_t(K_t \lambda_r - k_r + \lambda_r)}.
        $$
        А $\varphi_{4 \inf}$ остался неизменным. Таким образом:
        $$G_3^1 = G_2^1 \cap S_{\varphi_4}(G_2^1) = \set{
            \xin{2}{0}{P_0\left( 1 - \frac{\lambda_p}{k_p} \right)} \\
            \xin{3}{0}{\frac{k_r}{\lambda_r}} \\
            \xin{4}{0}{\frac{k_t k_r}{\lambda_t(K_t \lambda_r - k_r + \lambda_r)}}
        }.
        $$

        \item $\varphi_{1 \inf}$ и $\varphi_{1 \sup}$ не изменяются на $G_3^1$ из следующих соображений. Покуда $x_4$ не ограничен снизу числом, большим нуля, а $\varphi_{1}(x_4)$ на универсальном сечении --- убывающая функция, то (в силу вида $\varphi_{1}$) $\varphi_{1 \sup} = C_0$. Минимум же $= 0$, т.к. $\{x_1 = 0\}\,\subset S_{\varphi_1}$.
        
        Таким образом, 
        $$G_4^1 = G_3^1 \cap S_{\varphi_1}(G_3^1) = \set{
            \xin{1}{0}{ C_0 } \\
            \xin{2}{0}{P_0\left( 1 - \frac{\lambda_p}{k_p} \right)} \\
            \xin{3}{0}{\frac{k_r}{\lambda_r}} \\
            \xin{4}{0}{\frac{k_t k_r}{\lambda_t(K_t \lambda_r - k_r + \lambda_r)}}
        }.
        $$
    \end{enumerate}
    
    \item \textbf{Вторые 4 члена итерационной последовательности:}

    \begin{enumerate}
                
        \item  $\varphi_{2 \inf}(G_4^1) = 0\; (x_2 = 0),\; \varphi_{2 \sup}(G_4^1) = P_0\left( 1 - \frac{\lambda_p}{k_p} \right) \Rightarrow G_1^2 = G_4^1 $.

        \item $\varphi_{3 \sup}(G_1^2) =  \dfrac{k_r}{\lambda_r}, \; \varphi_{3 \inf}(G_1^2) = 
            \dfrac{k_r k_p}{\gamma_c C_0 k_p + \gamma_p P_0 (k_p - \lambda_p) + \lambda_r k_p}$;

            $$G_2^2 = G_1^2 \cap S_{\varphi_3}(G_1^2) = \set{
                \xin{1}{0}{ C_0 } \\
                \xin{2}{0}{P_0\left( 1 - \frac{\lambda_p}{k_p} \right)} \\
                \xin{3}{\frac{k_r k_p}{\gamma_c C_0 k_p + \gamma_p P_0 (k_p - \lambda_p) + \lambda_r k_p}}{\frac{k_r}{\lambda_r}} \\
                \xin{4}{0}{\frac{k_t k_r}{\lambda_t(K_t \lambda_r - k_r + \lambda_r)}}
            }.
            $$
        Это очень важный момент, поскольку мы наконец-то получили некоторое ограничение снизу для фазовых переменных, без которого многие оценки оказывались до этого бессмысленными.
        
        \item Опять воспользуемся тем, что ${x_4}'_{x_3} > 0: \varphi_{4 \sup}(G_2^2)$ остается тем же, $\varphi_{4 \inf}(G_2^2) = 
        x_4\left( \frac{k_r k_p}{\gamma_c C_0 k_p + \gamma_p P_0 (k_p - \lambda_p) + \lambda_r k_p} \right) \Rightarrow $
        $$ \varphi_{4 \inf}(G_2^2) = \dfrac{k_{p} k_{r} k_{t}}{\lambda_{t} \left(- k_{p} k_{r} + \left(K_{t} + 1\right) \left(C_{0} \gamma_{c} k_{p} + P_{0} \gamma_{p} \left(k_{p} - \lambda_{p}\right) + k_{p} \lambda_{r}\right)\right)}. \; \Rightarrow
        $$

        $$G_3^2 = G_2^2 \cap S_{\varphi_4}(G_2^2) = \set{
            \xin{1}{0}{ C_0 } \\
            \xin{2}{0}{P_0\left( 1 - \frac{\lambda_p}{k_p} \right)} \\
            \xin{3}{\frac{k_r k_p}{\gamma_c C_0 k_p + \gamma_p P_0 (k_p - \lambda_p) + \lambda_r k_p}}{\frac{k_r}{\lambda_r}} \\
            \xin{4}{\varphi_{4 \inf}(G_2^2)}{\frac{k_t k_r}{\lambda_t(K_t \lambda_r - k_r + \lambda_r)}}
            }.
        $$

        \item $\varphi_{1 \inf}(G_3^2) = 0$. Памятуя, что $\frac{\partial \varphi_1}{\partial x_2} > 0,\,  \frac{\partial \varphi_1}{\partial x_3} < 0,\,  \frac{\partial \varphi_1}{\partial x_4} < 0,$ установим  (если считать $ \varphi_1 = x_1(x_2, x_3, x_4)$ на $S_{\varphi_1}$): $ \varphi_{1 \sup}(G_3^2) =  x_1 \left( P_0 \left( 1 - \frac{\lambda_p}{k_p} \right), \varphi_{3 \inf}(G_1^2), \varphi_{1 \inf}(G_2^2) \right)$, а именно: 
        
        $$ \varphi_{1 \sup}(G_3^2) =  C_0 \dfrac{E^4}{D^4}, \text{ где:}$$
        \begin{multline*}
            E = \lambda_{t} \left(k_{p} k_{r} - \left(K_{c} + 1\right) \left(C_{0} \gamma_{c} k_{p} + P_{0} \gamma_{p} \left(k_{p} - \lambda_{p}\right) + k_{p} \lambda_{r}\right)\right) \cdot
            \\
            \cdot \left(k_{p} k_{r} - \left(K_{t} + 1\right) \left(C_{0} \gamma_{c} k_{p} + P_{0} \gamma_{p} \left(k_{p} - \lambda_{p}\right) + k_{p} \lambda_{r}\right)\right) \left(P_{0} \mu_{c} \left(k_{p} - \lambda_{p}\right) + k_{c} k_{p}\right);
        \end{multline*}
        \begin{multline*}
            D = \sqrt[4]{C_{0}} k_{p}^{2} k_{r} k_{t} \lambda_{c} \left(C_{0} \gamma_{c} k_{p} + P_{0} \gamma_{p} \left(k_{p} - \lambda_{p}\right) + k_{p} \lambda_{r}\right) + 
            \\
            +\lambda_{t} \left(k_{p} k_{r} - \left(K_{c} + 1\right) \left(C_{0} \gamma_{c} k_{p} + P_{0} \gamma_{p} \left(k_{p} - \lambda_{p}\right) + k_{p} \lambda_{r}\right)\right) \cdot
            \\
            \cdot \left(k_{p} k_{r} - \left(K_{t} + 1\right) \left(C_{0} \gamma_{c} k_{p} + P_{0} \gamma_{p} \left(k_{p} - \lambda_{p}\right) + k_{p} \lambda_{r}\right)\right) \left(P_{0} \mu_{c} \left(k_{p} - \lambda_{p}\right) + k_{c} k_{p}\right).
        \end{multline*}

        $$\text{Получили: } G_4^2 = G_3^2 \cap S_{\varphi_1}(G_3^2) = 
        \set{
            \xin{1}{\varphi_{1 \sup}(G_3^2)}{ C_0 } \\
            \xin{2}{0}{P_0\left( 1 - \frac{\lambda_p}{k_p} \right)} \\
            \xin{3}{\frac{k_r k_p}{\gamma_c C_0 k_p + \gamma_p P_0 (k_p - \lambda_p) + \lambda_r k_p}}{\frac{k_r}{\lambda_r}} \\
            \xin{4}{\varphi_{4 \inf}(G_2^2)}{\frac{k_t k_r}{\lambda_t(K_t \lambda_r - k_r + \lambda_r)}}
            }.
        $$

    \end{enumerate}


\end{enumerate}


\end{document}