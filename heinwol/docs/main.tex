\documentclass[11pt]{article}

    \usepackage[breakable]{tcolorbox}
    \usepackage{parskip} % Stop auto-indenting (to mimic markdown behaviour)
    
    \usepackage{iftex}
    \ifPDFTeX
    	\usepackage[T1]{fontenc}
    	\usepackage{mathpazo}
    \else
    	\usepackage{fontspec}
    \fi

    % Basic figure setup, for now with no caption control since it's done
    % automatically by Pandoc (which extracts ![](path) syntax from Markdown).
    \usepackage{graphicx}
    % Maintain compatibility with old templates. Remove in nbconvert 6.0
    \let\Oldincludegraphics\includegraphics
    % Ensure that by default, figures have no caption (until we provide a
    % proper Figure object with a Caption API and a way to capture that
    % in the conversion process - todo).
    \usepackage{caption}
    \DeclareCaptionFormat{nocaption}{}
    \captionsetup{format=nocaption,aboveskip=0pt,belowskip=0pt}

    \usepackage[Export]{adjustbox} % Used to constrain images to a maximum size
    \adjustboxset{max size={0.9\linewidth}{0.9\paperheight}}
    \usepackage{float}
    \floatplacement{figure}{H} % forces figures to be placed at the correct location
    \usepackage{xcolor} % Allow colors to be defined
    \usepackage{enumerate} % Needed for markdown enumerations to work
    \usepackage{geometry} % Used to adjust the document margins
    \usepackage{amsmath} % Equations
    \usepackage{amssymb} % Equations
    \usepackage{textcomp} % defines textquotesingle
    % Hack from http://tex.stackexchange.com/a/47451/13684:
    \AtBeginDocument{%
        \def\PYZsq{\textquotesingle}% Upright quotes in Pygmentized code
    }
    \usepackage{upquote} % Upright quotes for verbatim code
    \usepackage{eurosym} % defines \euro
    \usepackage[mathletters]{ucs} % Extended unicode (utf-8) support
    \usepackage{fancyvrb} % verbatim replacement that allows latex
    \usepackage{grffile} % extends the file name processing of package graphics 
                         % to support a larger range
    \makeatletter % fix for grffile with XeLaTeX
    \def\Gread@@xetex#1{%
      \IfFileExists{"\Gin@base".bb}%
      {\Gread@eps{\Gin@base.bb}}%
      {\Gread@@xetex@aux#1}%
    }
    \makeatother

    % The hyperref package gives us a pdf with properly built
    % internal navigation ('pdf bookmarks' for the table of contents,
    % internal cross-reference links, web links for URLs, etc.)
    \usepackage{hyperref}
    % The default LaTeX title has an obnoxious amount of whitespace. By default,
    % titling removes some of it. It also provides customization options.
    \usepackage{titling}
    \usepackage{longtable} % longtable support required by pandoc >1.10
    \usepackage{booktabs}  % table support for pandoc > 1.12.2
    \usepackage[inline]{enumitem} % IRkernel/repr support (it uses the enumerate* environment)
    \usepackage[normalem]{ulem} % ulem is needed to support strikethroughs (\sout)
                                % normalem makes italics be italics, not underlines
    \usepackage{mathrsfs}
    

    
    % Colors for the hyperref package
    \definecolor{urlcolor}{rgb}{0,.145,.698}
    \definecolor{linkcolor}{rgb}{.71,0.21,0.01}
    \definecolor{citecolor}{rgb}{.12,.54,.11}

    % ANSI colors
    \definecolor{ansi-black}{HTML}{3E424D}
    \definecolor{ansi-black-intense}{HTML}{282C36}
    \definecolor{ansi-red}{HTML}{E75C58}
    \definecolor{ansi-red-intense}{HTML}{B22B31}
    \definecolor{ansi-green}{HTML}{00A250}
    \definecolor{ansi-green-intense}{HTML}{007427}
    \definecolor{ansi-yellow}{HTML}{DDB62B}
    \definecolor{ansi-yellow-intense}{HTML}{B27D12}
    \definecolor{ansi-blue}{HTML}{208FFB}
    \definecolor{ansi-blue-intense}{HTML}{0065CA}
    \definecolor{ansi-magenta}{HTML}{D160C4}
    \definecolor{ansi-magenta-intense}{HTML}{A03196}
    \definecolor{ansi-cyan}{HTML}{60C6C8}
    \definecolor{ansi-cyan-intense}{HTML}{258F8F}
    \definecolor{ansi-white}{HTML}{C5C1B4}
    \definecolor{ansi-white-intense}{HTML}{A1A6B2}
    \definecolor{ansi-default-inverse-fg}{HTML}{FFFFFF}
    \definecolor{ansi-default-inverse-bg}{HTML}{000000}

    % commands and environments needed by pandoc snippets
    % extracted from the output of `pandoc -s`
    \providecommand{\tightlist}{%
      \setlength{\itemsep}{0pt}\setlength{\parskip}{0pt}}
    \DefineVerbatimEnvironment{Highlighting}{Verbatim}{commandchars=\\\{\}}
    % Add ',fontsize=\small' for more characters per line
    \newenvironment{Shaded}{}{}
    \newcommand{\KeywordTok}[1]{\textcolor[rgb]{0.00,0.44,0.13}{\textbf{{#1}}}}
    \newcommand{\DataTypeTok}[1]{\textcolor[rgb]{0.56,0.13,0.00}{{#1}}}
    \newcommand{\DecValTok}[1]{\textcolor[rgb]{0.25,0.63,0.44}{{#1}}}
    \newcommand{\BaseNTok}[1]{\textcolor[rgb]{0.25,0.63,0.44}{{#1}}}
    \newcommand{\FloatTok}[1]{\textcolor[rgb]{0.25,0.63,0.44}{{#1}}}
    \newcommand{\CharTok}[1]{\textcolor[rgb]{0.25,0.44,0.63}{{#1}}}
    \newcommand{\StringTok}[1]{\textcolor[rgb]{0.25,0.44,0.63}{{#1}}}
    \newcommand{\CommentTok}[1]{\textcolor[rgb]{0.38,0.63,0.69}{\textit{{#1}}}}
    \newcommand{\OtherTok}[1]{\textcolor[rgb]{0.00,0.44,0.13}{{#1}}}
    \newcommand{\AlertTok}[1]{\textcolor[rgb]{1.00,0.00,0.00}{\textbf{{#1}}}}
    \newcommand{\FunctionTok}[1]{\textcolor[rgb]{0.02,0.16,0.49}{{#1}}}
    \newcommand{\RegionMarkerTok}[1]{{#1}}
    \newcommand{\ErrorTok}[1]{\textcolor[rgb]{1.00,0.00,0.00}{\textbf{{#1}}}}
    \newcommand{\NormalTok}[1]{{#1}}
    
    % Additional commands for more recent versions of Pandoc
    \newcommand{\ConstantTok}[1]{\textcolor[rgb]{0.53,0.00,0.00}{{#1}}}
    \newcommand{\SpecialCharTok}[1]{\textcolor[rgb]{0.25,0.44,0.63}{{#1}}}
    \newcommand{\VerbatimStringTok}[1]{\textcolor[rgb]{0.25,0.44,0.63}{{#1}}}
    \newcommand{\SpecialStringTok}[1]{\textcolor[rgb]{0.73,0.40,0.53}{{#1}}}
    \newcommand{\ImportTok}[1]{{#1}}
    \newcommand{\DocumentationTok}[1]{\textcolor[rgb]{0.73,0.13,0.13}{\textit{{#1}}}}
    \newcommand{\AnnotationTok}[1]{\textcolor[rgb]{0.38,0.63,0.69}{\textbf{\textit{{#1}}}}}
    \newcommand{\CommentVarTok}[1]{\textcolor[rgb]{0.38,0.63,0.69}{\textbf{\textit{{#1}}}}}
    \newcommand{\VariableTok}[1]{\textcolor[rgb]{0.10,0.09,0.49}{{#1}}}
    \newcommand{\ControlFlowTok}[1]{\textcolor[rgb]{0.00,0.44,0.13}{\textbf{{#1}}}}
    \newcommand{\OperatorTok}[1]{\textcolor[rgb]{0.40,0.40,0.40}{{#1}}}
    \newcommand{\BuiltInTok}[1]{{#1}}
    \newcommand{\ExtensionTok}[1]{{#1}}
    \newcommand{\PreprocessorTok}[1]{\textcolor[rgb]{0.74,0.48,0.00}{{#1}}}
    \newcommand{\AttributeTok}[1]{\textcolor[rgb]{0.49,0.56,0.16}{{#1}}}
    \newcommand{\InformationTok}[1]{\textcolor[rgb]{0.38,0.63,0.69}{\textbf{\textit{{#1}}}}}
    \newcommand{\WarningTok}[1]{\textcolor[rgb]{0.38,0.63,0.69}{\textbf{\textit{{#1}}}}}
    
    
    % Define a nice break command that doesn't care if a line doesn't already
    % exist.
    \def\br{\hspace*{\fill} \\* }
    % Math Jax compatibility definitions
    \def\gt{>}
    \def\lt{<}
    \let\Oldtex\TeX
    \let\Oldlatex\LaTeX
    \renewcommand{\TeX}{\textrm{\Oldtex}}
    \renewcommand{\LaTeX}{\textrm{\Oldlatex}}
    % Document parameters
    % Document title
    \title{main}
    
    
    
    
    
% Pygments definitions
\makeatletter
\def\PY@reset{\let\PY@it=\relax \let\PY@bf=\relax%
    \let\PY@ul=\relax \let\PY@tc=\relax%
    \let\PY@bc=\relax \let\PY@ff=\relax}
\def\PY@tok#1{\csname PY@tok@#1\endcsname}
\def\PY@toks#1+{\ifx\relax#1\empty\else%
    \PY@tok{#1}\expandafter\PY@toks\fi}
\def\PY@do#1{\PY@bc{\PY@tc{\PY@ul{%
    \PY@it{\PY@bf{\PY@ff{#1}}}}}}}
\def\PY#1#2{\PY@reset\PY@toks#1+\relax+\PY@do{#2}}

\expandafter\def\csname PY@tok@w\endcsname{\def\PY@tc##1{\textcolor[rgb]{0.73,0.73,0.73}{##1}}}
\expandafter\def\csname PY@tok@c\endcsname{\let\PY@it=\textit\def\PY@tc##1{\textcolor[rgb]{0.25,0.50,0.50}{##1}}}
\expandafter\def\csname PY@tok@cp\endcsname{\def\PY@tc##1{\textcolor[rgb]{0.74,0.48,0.00}{##1}}}
\expandafter\def\csname PY@tok@k\endcsname{\let\PY@bf=\textbf\def\PY@tc##1{\textcolor[rgb]{0.00,0.50,0.00}{##1}}}
\expandafter\def\csname PY@tok@kp\endcsname{\def\PY@tc##1{\textcolor[rgb]{0.00,0.50,0.00}{##1}}}
\expandafter\def\csname PY@tok@kt\endcsname{\def\PY@tc##1{\textcolor[rgb]{0.69,0.00,0.25}{##1}}}
\expandafter\def\csname PY@tok@o\endcsname{\def\PY@tc##1{\textcolor[rgb]{0.40,0.40,0.40}{##1}}}
\expandafter\def\csname PY@tok@ow\endcsname{\let\PY@bf=\textbf\def\PY@tc##1{\textcolor[rgb]{0.67,0.13,1.00}{##1}}}
\expandafter\def\csname PY@tok@nb\endcsname{\def\PY@tc##1{\textcolor[rgb]{0.00,0.50,0.00}{##1}}}
\expandafter\def\csname PY@tok@nf\endcsname{\def\PY@tc##1{\textcolor[rgb]{0.00,0.00,1.00}{##1}}}
\expandafter\def\csname PY@tok@nc\endcsname{\let\PY@bf=\textbf\def\PY@tc##1{\textcolor[rgb]{0.00,0.00,1.00}{##1}}}
\expandafter\def\csname PY@tok@nn\endcsname{\let\PY@bf=\textbf\def\PY@tc##1{\textcolor[rgb]{0.00,0.00,1.00}{##1}}}
\expandafter\def\csname PY@tok@ne\endcsname{\let\PY@bf=\textbf\def\PY@tc##1{\textcolor[rgb]{0.82,0.25,0.23}{##1}}}
\expandafter\def\csname PY@tok@nv\endcsname{\def\PY@tc##1{\textcolor[rgb]{0.10,0.09,0.49}{##1}}}
\expandafter\def\csname PY@tok@no\endcsname{\def\PY@tc##1{\textcolor[rgb]{0.53,0.00,0.00}{##1}}}
\expandafter\def\csname PY@tok@nl\endcsname{\def\PY@tc##1{\textcolor[rgb]{0.63,0.63,0.00}{##1}}}
\expandafter\def\csname PY@tok@ni\endcsname{\let\PY@bf=\textbf\def\PY@tc##1{\textcolor[rgb]{0.60,0.60,0.60}{##1}}}
\expandafter\def\csname PY@tok@na\endcsname{\def\PY@tc##1{\textcolor[rgb]{0.49,0.56,0.16}{##1}}}
\expandafter\def\csname PY@tok@nt\endcsname{\let\PY@bf=\textbf\def\PY@tc##1{\textcolor[rgb]{0.00,0.50,0.00}{##1}}}
\expandafter\def\csname PY@tok@nd\endcsname{\def\PY@tc##1{\textcolor[rgb]{0.67,0.13,1.00}{##1}}}
\expandafter\def\csname PY@tok@s\endcsname{\def\PY@tc##1{\textcolor[rgb]{0.73,0.13,0.13}{##1}}}
\expandafter\def\csname PY@tok@sd\endcsname{\let\PY@it=\textit\def\PY@tc##1{\textcolor[rgb]{0.73,0.13,0.13}{##1}}}
\expandafter\def\csname PY@tok@si\endcsname{\let\PY@bf=\textbf\def\PY@tc##1{\textcolor[rgb]{0.73,0.40,0.53}{##1}}}
\expandafter\def\csname PY@tok@se\endcsname{\let\PY@bf=\textbf\def\PY@tc##1{\textcolor[rgb]{0.73,0.40,0.13}{##1}}}
\expandafter\def\csname PY@tok@sr\endcsname{\def\PY@tc##1{\textcolor[rgb]{0.73,0.40,0.53}{##1}}}
\expandafter\def\csname PY@tok@ss\endcsname{\def\PY@tc##1{\textcolor[rgb]{0.10,0.09,0.49}{##1}}}
\expandafter\def\csname PY@tok@sx\endcsname{\def\PY@tc##1{\textcolor[rgb]{0.00,0.50,0.00}{##1}}}
\expandafter\def\csname PY@tok@m\endcsname{\def\PY@tc##1{\textcolor[rgb]{0.40,0.40,0.40}{##1}}}
\expandafter\def\csname PY@tok@gh\endcsname{\let\PY@bf=\textbf\def\PY@tc##1{\textcolor[rgb]{0.00,0.00,0.50}{##1}}}
\expandafter\def\csname PY@tok@gu\endcsname{\let\PY@bf=\textbf\def\PY@tc##1{\textcolor[rgb]{0.50,0.00,0.50}{##1}}}
\expandafter\def\csname PY@tok@gd\endcsname{\def\PY@tc##1{\textcolor[rgb]{0.63,0.00,0.00}{##1}}}
\expandafter\def\csname PY@tok@gi\endcsname{\def\PY@tc##1{\textcolor[rgb]{0.00,0.63,0.00}{##1}}}
\expandafter\def\csname PY@tok@gr\endcsname{\def\PY@tc##1{\textcolor[rgb]{1.00,0.00,0.00}{##1}}}
\expandafter\def\csname PY@tok@ge\endcsname{\let\PY@it=\textit}
\expandafter\def\csname PY@tok@gs\endcsname{\let\PY@bf=\textbf}
\expandafter\def\csname PY@tok@gp\endcsname{\let\PY@bf=\textbf\def\PY@tc##1{\textcolor[rgb]{0.00,0.00,0.50}{##1}}}
\expandafter\def\csname PY@tok@go\endcsname{\def\PY@tc##1{\textcolor[rgb]{0.53,0.53,0.53}{##1}}}
\expandafter\def\csname PY@tok@gt\endcsname{\def\PY@tc##1{\textcolor[rgb]{0.00,0.27,0.87}{##1}}}
\expandafter\def\csname PY@tok@err\endcsname{\def\PY@bc##1{\setlength{\fboxsep}{0pt}\fcolorbox[rgb]{1.00,0.00,0.00}{1,1,1}{\strut ##1}}}
\expandafter\def\csname PY@tok@kc\endcsname{\let\PY@bf=\textbf\def\PY@tc##1{\textcolor[rgb]{0.00,0.50,0.00}{##1}}}
\expandafter\def\csname PY@tok@kd\endcsname{\let\PY@bf=\textbf\def\PY@tc##1{\textcolor[rgb]{0.00,0.50,0.00}{##1}}}
\expandafter\def\csname PY@tok@kn\endcsname{\let\PY@bf=\textbf\def\PY@tc##1{\textcolor[rgb]{0.00,0.50,0.00}{##1}}}
\expandafter\def\csname PY@tok@kr\endcsname{\let\PY@bf=\textbf\def\PY@tc##1{\textcolor[rgb]{0.00,0.50,0.00}{##1}}}
\expandafter\def\csname PY@tok@bp\endcsname{\def\PY@tc##1{\textcolor[rgb]{0.00,0.50,0.00}{##1}}}
\expandafter\def\csname PY@tok@fm\endcsname{\def\PY@tc##1{\textcolor[rgb]{0.00,0.00,1.00}{##1}}}
\expandafter\def\csname PY@tok@vc\endcsname{\def\PY@tc##1{\textcolor[rgb]{0.10,0.09,0.49}{##1}}}
\expandafter\def\csname PY@tok@vg\endcsname{\def\PY@tc##1{\textcolor[rgb]{0.10,0.09,0.49}{##1}}}
\expandafter\def\csname PY@tok@vi\endcsname{\def\PY@tc##1{\textcolor[rgb]{0.10,0.09,0.49}{##1}}}
\expandafter\def\csname PY@tok@vm\endcsname{\def\PY@tc##1{\textcolor[rgb]{0.10,0.09,0.49}{##1}}}
\expandafter\def\csname PY@tok@sa\endcsname{\def\PY@tc##1{\textcolor[rgb]{0.73,0.13,0.13}{##1}}}
\expandafter\def\csname PY@tok@sb\endcsname{\def\PY@tc##1{\textcolor[rgb]{0.73,0.13,0.13}{##1}}}
\expandafter\def\csname PY@tok@sc\endcsname{\def\PY@tc##1{\textcolor[rgb]{0.73,0.13,0.13}{##1}}}
\expandafter\def\csname PY@tok@dl\endcsname{\def\PY@tc##1{\textcolor[rgb]{0.73,0.13,0.13}{##1}}}
\expandafter\def\csname PY@tok@s2\endcsname{\def\PY@tc##1{\textcolor[rgb]{0.73,0.13,0.13}{##1}}}
\expandafter\def\csname PY@tok@sh\endcsname{\def\PY@tc##1{\textcolor[rgb]{0.73,0.13,0.13}{##1}}}
\expandafter\def\csname PY@tok@s1\endcsname{\def\PY@tc##1{\textcolor[rgb]{0.73,0.13,0.13}{##1}}}
\expandafter\def\csname PY@tok@mb\endcsname{\def\PY@tc##1{\textcolor[rgb]{0.40,0.40,0.40}{##1}}}
\expandafter\def\csname PY@tok@mf\endcsname{\def\PY@tc##1{\textcolor[rgb]{0.40,0.40,0.40}{##1}}}
\expandafter\def\csname PY@tok@mh\endcsname{\def\PY@tc##1{\textcolor[rgb]{0.40,0.40,0.40}{##1}}}
\expandafter\def\csname PY@tok@mi\endcsname{\def\PY@tc##1{\textcolor[rgb]{0.40,0.40,0.40}{##1}}}
\expandafter\def\csname PY@tok@il\endcsname{\def\PY@tc##1{\textcolor[rgb]{0.40,0.40,0.40}{##1}}}
\expandafter\def\csname PY@tok@mo\endcsname{\def\PY@tc##1{\textcolor[rgb]{0.40,0.40,0.40}{##1}}}
\expandafter\def\csname PY@tok@ch\endcsname{\let\PY@it=\textit\def\PY@tc##1{\textcolor[rgb]{0.25,0.50,0.50}{##1}}}
\expandafter\def\csname PY@tok@cm\endcsname{\let\PY@it=\textit\def\PY@tc##1{\textcolor[rgb]{0.25,0.50,0.50}{##1}}}
\expandafter\def\csname PY@tok@cpf\endcsname{\let\PY@it=\textit\def\PY@tc##1{\textcolor[rgb]{0.25,0.50,0.50}{##1}}}
\expandafter\def\csname PY@tok@c1\endcsname{\let\PY@it=\textit\def\PY@tc##1{\textcolor[rgb]{0.25,0.50,0.50}{##1}}}
\expandafter\def\csname PY@tok@cs\endcsname{\let\PY@it=\textit\def\PY@tc##1{\textcolor[rgb]{0.25,0.50,0.50}{##1}}}

\def\PYZbs{\char`\\}
\def\PYZus{\char`\_}
\def\PYZob{\char`\{}
\def\PYZcb{\char`\}}
\def\PYZca{\char`\^}
\def\PYZam{\char`\&}
\def\PYZlt{\char`\<}
\def\PYZgt{\char`\>}
\def\PYZsh{\char`\#}
\def\PYZpc{\char`\%}
\def\PYZdl{\char`\$}
\def\PYZhy{\char`\-}
\def\PYZsq{\char`\'}
\def\PYZdq{\char`\"}
\def\PYZti{\char`\~}
% for compatibility with earlier versions
\def\PYZat{@}
\def\PYZlb{[}
\def\PYZrb{]}
\makeatother


    % For linebreaks inside Verbatim environment from package fancyvrb. 
    \makeatletter
        \newbox\Wrappedcontinuationbox 
        \newbox\Wrappedvisiblespacebox 
        \newcommand*\Wrappedvisiblespace {\textcolor{red}{\textvisiblespace}} 
        \newcommand*\Wrappedcontinuationsymbol {\textcolor{red}{\llap{\tiny$\m@th\hookrightarrow$}}} 
        \newcommand*\Wrappedcontinuationindent {3ex } 
        \newcommand*\Wrappedafterbreak {\kern\Wrappedcontinuationindent\copy\Wrappedcontinuationbox} 
        % Take advantage of the already applied Pygments mark-up to insert 
        % potential linebreaks for TeX processing. 
        %        {, <, #, %, $, ' and ": go to next line. 
        %        _, }, ^, &, >, - and ~: stay at end of broken line. 
        % Use of \textquotesingle for straight quote. 
        \newcommand*\Wrappedbreaksatspecials {% 
            \def\PYGZus{\discretionary{\char`\_}{\Wrappedafterbreak}{\char`\_}}% 
            \def\PYGZob{\discretionary{}{\Wrappedafterbreak\char`\{}{\char`\{}}% 
            \def\PYGZcb{\discretionary{\char`\}}{\Wrappedafterbreak}{\char`\}}}% 
            \def\PYGZca{\discretionary{\char`\^}{\Wrappedafterbreak}{\char`\^}}% 
            \def\PYGZam{\discretionary{\char`\&}{\Wrappedafterbreak}{\char`\&}}% 
            \def\PYGZlt{\discretionary{}{\Wrappedafterbreak\char`\<}{\char`\<}}% 
            \def\PYGZgt{\discretionary{\char`\>}{\Wrappedafterbreak}{\char`\>}}% 
            \def\PYGZsh{\discretionary{}{\Wrappedafterbreak\char`\#}{\char`\#}}% 
            \def\PYGZpc{\discretionary{}{\Wrappedafterbreak\char`\%}{\char`\%}}% 
            \def\PYGZdl{\discretionary{}{\Wrappedafterbreak\char`\$}{\char`\$}}% 
            \def\PYGZhy{\discretionary{\char`\-}{\Wrappedafterbreak}{\char`\-}}% 
            \def\PYGZsq{\discretionary{}{\Wrappedafterbreak\textquotesingle}{\textquotesingle}}% 
            \def\PYGZdq{\discretionary{}{\Wrappedafterbreak\char`\"}{\char`\"}}% 
            \def\PYGZti{\discretionary{\char`\~}{\Wrappedafterbreak}{\char`\~}}% 
        } 
        % Some characters . , ; ? ! / are not pygmentized. 
        % This macro makes them "active" and they will insert potential linebreaks 
        \newcommand*\Wrappedbreaksatpunct {% 
            \lccode`\~`\.\lowercase{\def~}{\discretionary{\hbox{\char`\.}}{\Wrappedafterbreak}{\hbox{\char`\.}}}% 
            \lccode`\~`\,\lowercase{\def~}{\discretionary{\hbox{\char`\,}}{\Wrappedafterbreak}{\hbox{\char`\,}}}% 
            \lccode`\~`\;\lowercase{\def~}{\discretionary{\hbox{\char`\;}}{\Wrappedafterbreak}{\hbox{\char`\;}}}% 
            \lccode`\~`\:\lowercase{\def~}{\discretionary{\hbox{\char`\:}}{\Wrappedafterbreak}{\hbox{\char`\:}}}% 
            \lccode`\~`\?\lowercase{\def~}{\discretionary{\hbox{\char`\?}}{\Wrappedafterbreak}{\hbox{\char`\?}}}% 
            \lccode`\~`\!\lowercase{\def~}{\discretionary{\hbox{\char`\!}}{\Wrappedafterbreak}{\hbox{\char`\!}}}% 
            \lccode`\~`\/\lowercase{\def~}{\discretionary{\hbox{\char`\/}}{\Wrappedafterbreak}{\hbox{\char`\/}}}% 
            \catcode`\.\active
            \catcode`\,\active 
            \catcode`\;\active
            \catcode`\:\active
            \catcode`\?\active
            \catcode`\!\active
            \catcode`\/\active 
            \lccode`\~`\~ 	
        }
    \makeatother

    \let\OriginalVerbatim=\Verbatim
    \makeatletter
    \renewcommand{\Verbatim}[1][1]{%
        %\parskip\z@skip
        \sbox\Wrappedcontinuationbox {\Wrappedcontinuationsymbol}%
        \sbox\Wrappedvisiblespacebox {\FV@SetupFont\Wrappedvisiblespace}%
        \def\FancyVerbFormatLine ##1{\hsize\linewidth
            \vtop{\raggedright\hyphenpenalty\z@\exhyphenpenalty\z@
                \doublehyphendemerits\z@\finalhyphendemerits\z@
                \strut ##1\strut}%
        }%
        % If the linebreak is at a space, the latter will be displayed as visible
        % space at end of first line, and a continuation symbol starts next line.
        % Stretch/shrink are however usually zero for typewriter font.
        \def\FV@Space {%
            \nobreak\hskip\z@ plus\fontdimen3\font minus\fontdimen4\font
            \discretionary{\copy\Wrappedvisiblespacebox}{\Wrappedafterbreak}
            {\kern\fontdimen2\font}%
        }%
        
        % Allow breaks at special characters using \PYG... macros.
        \Wrappedbreaksatspecials
        % Breaks at punctuation characters . , ; ? ! and / need catcode=\active 	
        \OriginalVerbatim[#1,codes*=\Wrappedbreaksatpunct]%
    }
    \makeatother

    % Exact colors from NB
    \definecolor{incolor}{HTML}{303F9F}
    \definecolor{outcolor}{HTML}{D84315}
    \definecolor{cellborder}{HTML}{CFCFCF}
    \definecolor{cellbackground}{HTML}{F7F7F7}
    
    % prompt
    \makeatletter
    \newcommand{\boxspacing}{\kern\kvtcb@left@rule\kern\kvtcb@boxsep}
    \makeatother
    \newcommand{\prompt}[4]{
        \ttfamily\llap{{\color{#2}[#3]:\hspace{3pt}#4}}\vspace{-\baselineskip}
    }
    

    
    % Prevent overflowing lines due to hard-to-break entities
    \sloppy 
    % Setup hyperref package
    \hypersetup{
      breaklinks=true,  % so long urls are correctly broken across lines
      colorlinks=true,
      urlcolor=urlcolor,
      linkcolor=linkcolor,
      citecolor=citecolor,
      }
    % Slightly bigger margins than the latex defaults
    
    \geometry{verbose,tmargin=1in,bmargin=1in,lmargin=1in,rmargin=1in}
    
    

\begin{document}
    
    \maketitle
    
    

    
    
    \begin{verbatim}
<IPython.core.display.HTML object>
    \end{verbatim}

    
    $\displaystyle \left[\begin{matrix}- \frac{\lambda_{c} x_{1} x_{4}}{K_{c} - x_{3} + 1} + x_{1}^{\frac{3}{4}} \left(1 - \sqrt[4]{\frac{x_{1}}{C_{0}}}\right) \left(k_{c} + \mu_{c} x_{2}\right)\\- \lambda_{p} x_{2} + x_{2} \left(1 - \frac{x_{2}}{P_{0}}\right) \left(k_{p} + \frac{\mu_{p} x_{1}}{K_{p} + x_{1}}\right)\\k_{r} - x_{3} \left(\gamma_{c} x_{1} + \gamma_{p} x_{2} + \lambda_{r}\right)\\\frac{k_{t} x_{3}}{K_{t} - x_{3} + 1} - \lambda_{t} x_{4}\end{matrix}\right]$

    
    Докажем инвариантность первого октанта, для этого рассмотрим отдельно
все координатные плоскости.

Положим \(x_1 = 0:\)

    F\_new = F.subs(x{[}1{]}, 0) display(F\_new)

    \(x_2 = 0:\)

    F\_new = F.subs(x{[}2{]}, 0) display(F\_new)

    \(x_3 = 0:\)

    F\_new = F.subs(x{[}3{]}, 0) display(F\_new)

    \(x_4 = 0:\)

    F\_new = F.subs(x{[}4{]}, 0) display(F\_new)

    \hypertarget{section}{%
\subsection{----------------------------------------------}\label{section}}

\hypertarget{ux43dux430ux445ux43eux436ux434ux435ux43dux438ux435-ux43fux43eux43bux43eux436ux435ux43dux438ux439-ux440ux430ux432ux43dux43eux432ux435ux441ux438ux44f}{%
\section{Нахождение положений
равновесия:}\label{ux43dux430ux445ux43eux436ux434ux435ux43dux438ux435-ux43fux43eux43bux43eux436ux435ux43dux438ux439-ux440ux430ux432ux43dux43eux432ux435ux441ux438ux44f}}

    $\displaystyle \left[\begin{matrix}- \frac{\lambda_{c} x_{1} x_{4}}{K_{c} - x_{3} + 1} + x_{1}^{\frac{3}{4}} \left(1 - \sqrt[4]{\frac{x_{1}}{C_{0}}}\right) \left(k_{c} + \mu_{c} x_{2}\right) = 0\\- \lambda_{p} x_{2} + x_{2} \left(1 - \frac{x_{2}}{P_{0}}\right) \left(k_{p} + \frac{\mu_{p} x_{1}}{K_{p} + x_{1}}\right) = 0\\k_{r} - x_{3} \left(\gamma_{c} x_{1} + \gamma_{p} x_{2} + \lambda_{r}\right) = 0\\\frac{k_{t} x_{3}}{K_{t} - x_{3} + 1} - \lambda_{t} x_{4} = 0\end{matrix}\right]$

    
    \hypertarget{ux43dux430ux439ux434ux435ux43c-ux43fux43eux43bux43eux436ux435ux43dux438ux44f-ux440ux430ux432ux43dux43eux432ux435ux441ux438ux44f-ux432-ux43aux43eux43eux440ux434ux438ux43dux430ux442ux43dux44bux445-ux43fux43bux43eux441ux43aux43eux441ux442ux44fux445-ux438-x_3-1}{%
\subsubsection{\texorpdfstring{Найдем положения равновесия в
координатных плоскостях и
\(\{x_3 = 1\}\)}{Найдем положения равновесия в координатных плоскостях и \textbackslash\{x\_3 = 1\textbackslash\}}}\label{ux43dux430ux439ux434ux435ux43c-ux43fux43eux43bux43eux436ux435ux43dux438ux44f-ux440ux430ux432ux43dux43eux432ux435ux441ux438ux44f-ux432-ux43aux43eux43eux440ux434ux438ux43dux430ux442ux43dux44bux445-ux43fux43bux43eux441ux43aux43eux441ux442ux44fux445-ux438-x_3-1}}

    $x_1 = 0:$

    
    $\displaystyle \left[\begin{matrix}\text{True}\\k_{p} x_{2} \left(1 - \frac{x_{2}}{P_{0}}\right) - \lambda_{p} x_{2} = 0\\k_{r} - x_{3} \left(\gamma_{p} x_{2} + \lambda_{r}\right) = 0\\\frac{k_{t} x_{3}}{K_{t} - x_{3} + 1} - \lambda_{t} x_{4} = 0\end{matrix}\right]$

    
    \(\phantom{tab}\) а)
\(x_2 = 0 \Rightarrow (3) \Rightarrow x_3 = \dfrac{k_r}{\lambda_r} < 1, \; \text{по крайней мере, если подставить значения из приложения.}\)
\$ \text{Тогда}; (4) \Rightarrow x\_4 =
\dfrac{k_t k_r}{\lambda_t(\lambda_r K_t + \lambda_r - k_r)}. \text{}\$
\(\phantom{tab}\) б) \(x_2 \neq 0 \Rightarrow\)

    $(2) \Rightarrow x_2 = \frac{P_{0} \left(k_{p} - \lambda_{p}\right)}{k_{p}} \;\; (*);$

    
    $(3),(*) \Rightarrow x_3 = \frac{k_{p} k_{r}}{P_{0} \gamma_{p} \left(k_{p} - \lambda_{p}\right) + k_{p} \lambda_{r}} \;\; (**);$

    
    $(4),(**) \Rightarrow x_3 = \frac{k_{p} k_{r} k_{t}}{\lambda_{t} \left(- k_{p} k_{r} + \left(K_{t} + 1\right) \left(P_{0} \gamma_{p} \left(k_{p} - \lambda_{p}\right) + k_{p} \lambda_{r}\right)\right)}.$

    
    \begin{Verbatim}[commandchars=\\\{\}]
Мы нашли все 2 положения равновесия в первой координатной плоскости.
    \end{Verbatim}

    $x_2 = 0:$

    
    $\displaystyle \left[\begin{matrix}k_{c} x_{1}^{\frac{3}{4}} \left(1 - \sqrt[4]{\frac{x_{1}}{C_{0}}}\right) - \frac{\lambda_{c} x_{1} x_{4}}{K_{c} - x_{3} + 1} = 0\\\text{True}\\k_{r} - x_{3} \left(\gamma_{c} x_{1} + \lambda_{r}\right) = 0\\\frac{k_{t} x_{3}}{K_{t} - x_{3} + 1} - \lambda_{t} x_{4} = 0\end{matrix}\right]$

    
    $(3) \Rightarrow x_3 = \frac{k_{r}}{\gamma_{c} x_{1} + \lambda_{r}} \;\; (*);$

    
    $(4),(*) \Rightarrow x_4 = \frac{k_{r} k_{t}}{\lambda_{t} \left(- k_{r} + \left(K_{t} + 1\right) \left(\gamma_{c} x_{1} + \lambda_{r}\right)\right)};$

    
    \begin{Verbatim}[commandchars=\\\{\}]
Подставив в (1), получим:
    \end{Verbatim}

    $\displaystyle k_{c} x_{1}^{\frac{3}{4}} \left(1 - \sqrt[4]{\frac{x_{1}}{C_{0}}}\right) - \frac{k_{r} k_{t} \lambda_{c} x_{1}}{\lambda_{t} \left(- k_{r} + \left(K_{t} + 1\right) \left(\gamma_{c} x_{1} + \lambda_{r}\right)\right) \left(K_{c} - \frac{k_{r}}{\gamma_{c} x_{1} + \lambda_{r}} + 1\right)} = 0$

    
    Полагая \(x_1 \neq 0\) (данный случай уже рассмотрен) , разделим на
\(x_1^{\frac{3}{4}}\) Затем, осуществив замену \(x_1 = y^4\), а также
приведя все дроби к общему знаменателю, получим эквивалентное уравнение:

    $- \sqrt[4]{C_{0}} k_{r} k_{t} \lambda_{c} y \left(\gamma_{c} y^{4} + \lambda_{r}\right) + k_{c} \lambda_{t} \left(\sqrt[4]{C_{0}} - y\right) \left(- k_{r} + \left(K_{c} + 1\right) \left(\gamma_{c} y^{4} + \lambda_{r}\right)\right) \left(- k_{r} + \left(K_{t} + 1\right) \left(\gamma_{c} y^{4} + \lambda_{r}\right)\right) = 0$

    
    $x_3 = 0:$

    
    $\displaystyle \left[\begin{matrix}- \frac{\lambda_{c} x_{1} x_{4}}{K_{c} + 1} + x_{1}^{\frac{3}{4}} \left(1 - \sqrt[4]{\frac{x_{1}}{C_{0}}}\right) \left(k_{c} + \mu_{c} x_{2}\right) = 0\\- \lambda_{p} x_{2} + x_{2} \left(1 - \frac{x_{2}}{P_{0}}\right) \left(k_{p} + \frac{\mu_{p} x_{1}}{K_{p} + x_{1}}\right) = 0\\k_{r} = 0\\- \lambda_{t} x_{4} = 0\end{matrix}\right]$

    
    \begin{Verbatim}[commandchars=\\\{\}]
То есть нет решений в 3-ей координатной плоскости
    \end{Verbatim}

    $x_4 = 0:$

    
    $\displaystyle \left[\begin{matrix}x_{1}^{\frac{3}{4}} \left(1 - \sqrt[4]{\frac{x_{1}}{C_{0}}}\right) \left(k_{c} + \mu_{c} x_{2}\right) = 0\\- \lambda_{p} x_{2} + x_{2} \left(1 - \frac{x_{2}}{P_{0}}\right) \left(k_{p} + \frac{\mu_{p} x_{1}}{K_{p} + x_{1}}\right) = 0\\k_{r} - x_{3} \left(\gamma_{c} x_{1} + \gamma_{p} x_{2} + \lambda_{r}\right) = 0\\\frac{k_{t} x_{3}}{K_{t} - x_{3} + 1} = 0\end{matrix}\right]$

    
    \begin{Verbatim}[commandchars=\\\{\}]
Аналогично, нет решений в 4-ой координатной плоскости
    \end{Verbatim}

    $x_3 = 1:$

    
    $\displaystyle \left[\begin{matrix}x_{1}^{\frac{3}{4}} \left(1 - \sqrt[4]{\frac{x_{1}}{C_{0}}}\right) \left(k_{c} + \mu_{c} x_{2}\right) - \frac{\lambda_{c} x_{1} x_{4}}{K_{c}} = 0\\- \lambda_{p} x_{2} + x_{2} \left(1 - \frac{x_{2}}{P_{0}}\right) \left(k_{p} + \frac{\mu_{p} x_{1}}{K_{p} + x_{1}}\right) = 0\\- \gamma_{c} x_{1} - \gamma_{p} x_{2} + k_{r} - \lambda_{r} = 0\\- \lambda_{t} x_{4} + \frac{k_{t}}{K_{t}} = 0\end{matrix}\right]$

    
    Второе уравнение:

    
    $\displaystyle - \lambda_{p} x_{2} + x_{2} \left(1 - \frac{x_{2}}{P_{0}}\right) \left(k_{p} + \frac{\mu_{p} x_{1}}{K_{p} + x_{1}}\right) = 0$

    
    $\displaystyle \frac{x_{2} \left(- P_{0} \lambda_{p} \left(K_{p} + x_{1}\right) + \left(P_{0} - x_{2}\right) \left(k_{p} \left(K_{p} + x_{1}\right) + \mu_{p} x_{1}\right)\right)}{P_{0} \left(K_{p} + x_{1}\right)} = 0$

    
    а) Положим $x_2 = 0$, поскольку решение аналогичной задачи поиска ПР в плоскости $x_2=0$         привела к многочлену 9-й степени и пока не была решена

    
    $\displaystyle \left[\begin{matrix}k_{c} x_{1}^{\frac{3}{4}} \left(1 - \sqrt[4]{\frac{x_{1}}{C_{0}}}\right) - \frac{\lambda_{c} x_{1} x_{4}}{K_{c}} = 0\\\text{True}\\- \gamma_{c} x_{1} + k_{r} - \lambda_{r} = 0\\- \lambda_{t} x_{4} + \frac{k_{t}}{K_{t}} = 0\end{matrix}\right]$

    
    Выразим $x_1$ и $x_4$ из третьего и четвертого уравнений:

    
    $\displaystyle x_{1} = \frac{k_{r} - \lambda_{r}}{\gamma_{c}}$

    
    $\displaystyle x_{4} = \frac{k_{t}}{K_{t} \lambda_{t}}$

    
    Проверим, выполняется ли при этом первое уравнение

    
    $\displaystyle - \frac{K_{c} K_{t} \gamma_{c} k_{c} \lambda_{t} \left(\frac{k_{r} - \lambda_{r}}{\gamma_{c}}\right)^{\frac{3}{4}} \left(\sqrt[4]{\frac{k_{r} - \lambda_{r}}{C_{0} \gamma_{c}}} - 1\right) + k_{t} \lambda_{c} \left(k_{r} - \lambda_{r}\right)}{K_{c} K_{t} \gamma_{c} \lambda_{t}}$

    
    б) Пусть $x_2 \neq 0$. Тогда разделим на $x_2$ второе уравнение системы $F_{\big|x_3 = 1}$

    
    $\displaystyle \left[\begin{matrix}x_{1}^{\frac{3}{4}} \left(1 - \sqrt[4]{\frac{x_{1}}{C_{0}}}\right) \left(k_{c} + \mu_{c} x_{2}\right) - \frac{\lambda_{c} x_{1} x_{4}}{K_{c}} = 0\\\frac{- P_{0} \lambda_{p} \left(K_{p} + x_{1}\right) + \left(P_{0} - x_{2}\right) \left(k_{p} \left(K_{p} + x_{1}\right) + \mu_{p} x_{1}\right)}{P_{0} \left(K_{p} + x_{1}\right)} = 0\\- \gamma_{c} x_{1} - \gamma_{p} x_{2} + k_{r} - \lambda_{r} = 0\\- \lambda_{t} x_{4} + \frac{k_{t}}{K_{t}} = 0\end{matrix}\right]$

    
    $\displaystyle \left[\begin{matrix}x_{1}^{\frac{3}{4}} \left(1 - \sqrt[4]{\frac{x_{1}}{C_{0}}}\right) \left(k_{c} + \mu_{c} x_{2}\right) - \frac{k_{t} \lambda_{c} x_{1}}{K_{c} K_{t} \lambda_{t}} = 0\\- P_{0} \lambda_{p} \left(K_{p} + x_{1}\right) + \left(P_{0} - x_{2}\right) \left(k_{p} \left(K_{p} + x_{1}\right) + \mu_{p} x_{1}\right) = 0\\- \gamma_{c} x_{1} - \gamma_{p} x_{2} + k_{r} - \lambda_{r} = 0\end{matrix}\right]$

    
    Система не имеет решений. Из первого уравнения $x_2$ выражается как:  $x_2 = \phi(x_1^{\frac{3}{4}})$

    
    Видно, что это не удовлетворяет последнему уравнению системы. Таким образом, нет ПР в плоскости $x_3 = 1$

    
    \hypertarget{ux43aux43eux43eux440ux434ux438ux43dux430ux442ux43dux44bux435-ux444ux443ux43dux43aux446ux438ux438}{%
\subsection{Координатные
функции}\label{ux43aux43eux43eux440ux434ux438ux43dux430ux442ux43dux44bux435-ux444ux443ux43dux43aux446ux438ux438}}

    Рассмотрим координатные функции вида
\(\varphi_i = x_i\,,\; i = \overline{1,4}\) Вернемся для этого к
уравнению \(F(x_1,x_2,x_3,x_4) = 0:\)

    $\displaystyle \left[\begin{matrix}- \frac{\lambda_{c} x_{1} x_{4}}{K_{c} - x_{3} + 1} + x_{1}^{\frac{3}{4}} \left(1 - \sqrt[4]{\frac{x_{1}}{C_{0}}}\right) \left(k_{c} + \mu_{c} x_{2}\right) = 0\\- \lambda_{p} x_{2} + x_{2} \left(1 - \frac{x_{2}}{P_{0}}\right) \left(k_{p} + \frac{\mu_{p} x_{1}}{K_{p} + x_{1}}\right) = 0\\k_{r} - x_{3} \left(\gamma_{c} x_{1} + \gamma_{p} x_{2} + \lambda_{r}\right) = 0\\\frac{k_{t} x_{3}}{K_{t} - x_{3} + 1} - \lambda_{t} x_{4} = 0\end{matrix}\right]$

    
    Для первой координатной функции имеем:
\[ S_{\varphi_1} = \{ x_1 = 0\}\, \cup \, 
\left\{ x_1 = 
\frac{C_{0} \alpha^{4}{\left(x_{2},x_{3} \right)}}{\left(\sqrt[4]{C_{0}} \lambda_{c} x_{4} + \alpha{\left(x_{2},x_{3} \right)}\right)^{4}}
\right\} \]

    $$ \text{Явное выражение для } x_1: $$

    
    $ x_1 = \frac{C_{0} \alpha^{4}{\left(x_{2},x_{3} \right)}}{\left(\sqrt[4]{C_{0}} \lambda_{c} x_{4} + \alpha{\left(x_{2},x_{3} \right)}\right)^{4}}$, где $\alpha = \left(k_{c} + \mu_{c} x_{2}\right) \left(K_{c} - x_{3} + 1\right) > 0$

    
    $\nabla \varphi_1 = \left[\begin{matrix}\frac{4 C_{0}^{\frac{5}{4}} \lambda_{c} \mu_{c} x_{4} \left(k_{c} + \mu_{c} x_{2}\right)^{3} \left(K_{c} - x_{3} + 1\right)^{4}}{\left(\sqrt[4]{C_{0}} \lambda_{c} x_{4} + \left(k_{c} + \mu_{c} x_{2}\right) \left(K_{c} - x_{3} + 1\right)\right)^{5}} & - \frac{4 C_{0}^{\frac{5}{4}} \lambda_{c} x_{4} \left(k_{c} + \mu_{c} x_{2}\right)^{4} \left(K_{c} - x_{3} + 1\right)^{3}}{\left(\sqrt[4]{C_{0}} \lambda_{c} x_{4} + \left(k_{c} + \mu_{c} x_{2}\right) \left(K_{c} - x_{3} + 1\right)\right)^{5}} & - \frac{4 C_{0}^{\frac{5}{4}} \lambda_{c} \left(k_{c} + \mu_{c} x_{2}\right)^{4} \left(K_{c} - x_{3} + 1\right)^{4}}{\left(\sqrt[4]{C_{0}} \lambda_{c} x_{4} + \left(k_{c} + \mu_{c} x_{2}\right) \left(K_{c} - x_{3} + 1\right)\right)^{5}}\end{matrix}\right] $

    
    Отсюда видно, что градиент ни в какой точке не обращается в нуль, потому
что любая компонента нигде не ноль, поскольку \(\alpha > 0.\)
Конкретнее,
\(\dfrac{\partial \varphi_1}{\partial x_2} > 0,\,  \dfrac{\partial \varphi_1}{\partial x_3} < 0,\,  \dfrac{\partial \varphi_1}{\partial x_4} < 0.\)
Очевидно, \(\varphi_{3 \min} = 0\) (при \(x_1 = 0\)). Для установления
супремума воспользуемся знаниями частных производных: значение будет тем
больше, чем меньше \(x_3\) и \(x_4\) и чем больше \(x_2\). Положив
\(x_3 = x_4 = 0\), получим
\(x_1 = C_0 \Rightarrow \varphi_{1 \max} = C_0\)

    Для второй координатной функции:
\[ S_{\varphi_2} = \{ x_2 = 0\}\, \cup \, 
\left\{ x_2 = 
\dfrac{P_{0} \left(K_{p} k_{p} - K_{p} \lambda_{p} + x_{1} \left(k_{p} - \lambda_{p} + \mu_{p}\right)\right)}{- K_{p} k_{p} + x_{1} \left(k_{p} + \mu_{p}\right)}
\right\} \]

    $ \text{Явное выражение для } x_2: $

    
    $$ x_2 = \frac{P_{0} \left(K_{p} k_{p} - K_{p} \lambda_{p} + x_{1} \left(k_{p} - \lambda_{p} + \mu_{p}\right)\right)}{- K_{p} k_{p} + x_{1} \left(k_{p} + \mu_{p}\right)}$$

    
    $ x_2' = - \frac{P_{0} \left(\left(k_{p} + \mu_{p}\right) \left(K_{p} k_{p} - K_{p} \lambda_{p} + x_{1} \left(k_{p} - \lambda_{p} + \mu_{p}\right)\right) + \left(K_{p} k_{p} - x_{1} \left(k_{p} + \mu_{p}\right)\right) \left(k_{p} - \lambda_{p} + \mu_{p}\right)\right)}{\left(K_{p} k_{p} - x_{1} \left(k_{p} + \mu_{p}\right)\right)^{2}} $

    
    $ \text{Явное выражение для } x_3: $

    
    $$ x_3 = \frac{k_{r}}{\gamma_{c} x_{1} + \gamma_{p} x_{2} + \lambda_{r}}$$

    
    $\nabla x_3 = \left[\begin{matrix}- \frac{\gamma_{c} k_{r}}{\left(\gamma_{c} x_{1} + \gamma_{p} x_{2} + \lambda_{r}\right)^{2}} & - \frac{\gamma_{p} k_{r}}{\left(\gamma_{c} x_{1} + \gamma_{p} x_{2} + \lambda_{r}\right)^{2}}\end{matrix}\right] $

    
    Очевидно, что никогда не выполняется \(\nabla x_3 = 0\,\) (раз все \(x\)
неотрицательны), так что функция \(\varphi_3(\vec{x}) = x_3\) достигает
своих экстремальных значений (на универсальном сечении) лишь на границе,
либо не достигает их вовсе. Действительно, видно, что достигается
максимум при \(x_1 = x_2 = 0,\) при этом
\(\varphi_{3 \max} = \dfrac{k_r}{\lambda_r}\) А вот минимального
значения \(\varphi_3\) не достигает, но \(\varphi_{3 \inf} = 0\),
очевидно.

    $ \text{Явное выражение для } x_4: $

    
    $$ x_4 = \frac{k_{t} x_{3}}{\lambda_{t} \left(K_{t} - x_{3} + 1\right)}$$

    
    $x_4' = \frac{k_{t} x_{3}}{\lambda_{t} \left(K_{t} - x_{3} + 1\right)^{2}} + \frac{k_{t}}{\lambda_{t} \left(K_{t} - x_{3} + 1\right)} $

    
    Приходим к выводу, что \(\varphi_{4 \min} = 0\) (достигается при
\(x_3 = 0\)), а максимум достигается при \$x\_3 = 1,\varphi\_\{4 \max\}
= \dfrac{k_t}{\lambda_t K_t} \$

    \hypertarget{ux43fux43eux441ux442ux440ux43eux435ux43dux438ux435-ux43fux43eux441ux43bux435ux434ux43eux432ux430ux442ux435ux43bux44cux43dux43eux441ux442ux438-ux43bux43eux43aux430ux43bux438ux437ux443ux44eux449ux438ux445-ux444ux443ux43dux43aux446ux438ux439-ux438-ux43cux43dux43eux436ux435ux441ux442ux432}{%
\subsection{Построение последовательности локализующих функций и
множеств}\label{ux43fux43eux441ux442ux440ux43eux435ux43dux438ux435-ux43fux43eux441ux43bux435ux434ux43eux432ux430ux442ux435ux43bux44cux43dux43eux441ux442ux438-ux43bux43eux43aux430ux43bux438ux437ux443ux44eux449ux438ux445-ux444ux443ux43dux43aux446ux438ux439-ux438-ux43cux43dux43eux436ux435ux441ux442ux432}}


    % Add a bibliography block to the postdoc
    
    
    
\end{document}
